\documentclass[12pt]{article}

\usepackage[english]{babel}
\usepackage[utf8x]{inputenc}
\usepackage{amsmath}
\usepackage{enumitem}
\usepackage{graphicx}
\usepackage{ulem}
\usepackage{caption}
\usepackage{placeins}
\usepackage[usenames,dvipsnames]{color}
\usepackage[colorinlistoftodos]{todonotes}
\usepackage{listings}
\usepackage{fixltx2e}
\usepackage{scrpage2}
\usepackage{lastpage}
\clearscrheadfoot
\pagestyle{scrheadings}
\usepackage{glossaries}
\usepackage[
top    = 2.75cm,
bottom = 2.00cm,
left   = 2.50cm,
right  = 2.00cm]{geometry}
\setcounter{secnumdepth}{4}
\definecolor{dkgreen}{rgb}{0,0.6,0}
\definecolor{gray}{rgb}{0.5,0.5,0.5}
\definecolor{mauve}{rgb}{0.58,0,0.82}

\newcommand{\executeiffilenewer}[3]{%
\ifnum\pdfstrcmp{\pdffilemoddate{#1}}%
{\pdffilemoddate{#2}}>0%
{\immediate\write18{#3}}\fi%
}
\newcommand{\includesvg}[1]{%
\executeiffilenewer{#1.svg}{#1.pdf}%
{inkscape -z -D --file=#1.pdf --export-pdf=#1.pdf --export-latex}%
\input{#1.pdf_tex}%
}

\lstset{frame=tb,
  language=Java,
  aboveskip=3mm,
  belowskip=3mm,
  showstringspaces=false,
  columns=flexible,
  basicstyle={\small\ttfamily},
  numbers=none,
  numberstyle=\tiny\color{gray},
  keywordstyle=\color{blue},
  commentstyle=\color{dkgreen},
  stringstyle=\color{mauve},
  breaklines=true,
  breakatwhitespace=true
  tabsize=3
}

\begin{document}
\begin{titlepage}
\begin{center}
% Oberer Teil der Titelseite:
\includegraphics[width=0.5\textwidth]{images/logo}\\[1cm]    

\LARGE TGM - HTBLuVA Wien XX \\ IT Department  \\[1.5cm]

% Title
\rule{1.0\textwidth}{1mm}
{ \huge \bfseries \\[0.4cm]  \huge Service Oriented Architecture and Restful Webservice \\ \LARGE Dezsys 08  \\[0.4cm] }

\rule{1.0\textwidth}{1mm}



% Author and supervisor
\noindent 
\vspace{5cm}

\begin{center}
\large
Authors: 
Siegel \textsc{Hannah} \&
Nachname2 \textsc{Vorname2}
\end{center}

\vfill

% Bottom of the page
{\large \today}

\end{center}
\end{titlepage}

\tableofcontents


%HEADER AND FOOTER
\pagenumbering{arabic}
\ohead{\headmark}
\automark{section}
\ifoot{© Authors}
\ofoot{\pagemark ~of \pageref{LastPage}}

\newpage



\section{Working time}
\begin{center}
\end{center}
\begin{table}[h]
\begin{tabular}{|p{0.5\textwidth}|p{0.1\textwidth}|p{0.2\textwidth}|p{0.2\textwidth}|}
\hline
\textbf{Task}    & \textbf{Person}   & \textbf{Estimated}  & \textbf{Final}\\ \hline \hline

Preparation and Frameworks & Haidn &  60 minutes & 60 minutes  \\ \hline 


Creating the Knowledge base with Hibernate & Haidn &  60   minutes & \textcolor{red}{180 minutes}
   \\ \hline 
Creating the Knowledge base & Haidn &  &  \textcolor{red}{60 minutes} \\ \hline 

Inserting Test-Data  & Haidn &  30 minutes  &  30 minutes  \\ \hline 

Testing the Performance  & Haidn &  30  minutes  &  60  minutes  \\ \hline 

RestFul CRUD Operations & Siegel &  90   minutes & \textcolor{red}{210 minutes} \\  
  & Haidn &  90 minutes  &  90   minutes  \\ \hline 


Generate SOA Webservice & Siegel &  60 minutes &  \textcolor{red}{120 minutes} \\ \hline 

Include SOAP for SOA Webservice & Siegel &  30 minutes &  60 minutes \\ \hline 
Generate WSDL File & Siegel &  30 minutes & \textcolor{green}{10 minutes}  \\ \hline 

Generate Website & Haidn &  60 minutes &  90 minutes \\ \hline 

Generate Client for SOA Webservice & Siegel &  60 minutes & \textcolor{red}{180 minutes} \\ \hline 
Document Datatransfer with SOAP & Siegel &  60 minutes &  60 minutes \\ \hline 

Testing & Siegel &  90  minutes  &  90  minutes  \\  
  & Haidn &  90  minutes  &  90  minutes   \\ \hline 


Documentation  & Siegel &  60 minutes  &  60 minutes  \\ \hline 
\hline 

Testing & Siegel &  8 hours  & xxx   \\  
  & Haidn &  7 hours   & xxx   \\ \hline 


 \hline 
\textbf{Total Team} & & \textbf{15 hours} &\textbf{xx hours}  \\ \hline 
\end{tabular}
\end{table}

\section{Aufgabenstellung}

\section{Knowledge Base}
For the Knowledge Base we decided to use Hibernate and JPA. \\
We were using IntelliJ and therefore it was quite easy. \\
At first, we were making a new project, which has already a Hello World Webservice. We copied the hibernate config file from the \textit{Westbahn} project into the \texttt{src} folder.\\
Also, we had to add all the needed libraries, such as the mysql connector and the hibernate libraries.\\
\\
The Knowledge Base class can be seen in Listing \ref{KnowledgeBase}.
\begin{lstlisting}[caption=KnowledgeBase entity class, label=KnowledgeBase]
// import statements

@Entity
public class KnowledgeBase {
    @Id
    @GeneratedValue
    private Long ID;

    @Column(unique=true)
    @Size(min=10, max=500)
    private String text;

    @Column(unique=true)
    @Size(min=10, max=50)
    private String topic;

	// constructor, getter and setter
}
\end{lstlisting}

- database connection is not working tho
-> mocking it
- new module : JPA module... (file - new module - JPA, Hibernate, SQL Supp)


brauch nun schon 2 stunden fuer hibernate
skipping this


\section{Build a RESTful WebService}
%"RESTful web services are built to work best on the Web. Representational State Transfer (REST) is an architectural style that specifies constraints, such as the uniform interface, that if applied to a web service induce desirable properties, such as performance, scalability, and modifiability, that enable services to work best on the Web. In the REST architectural style, data and functionality are considered resources and are accessed using Uniform Resource Identifiers (URIs), typically links on the Web. The resources are acted upon by using a set of simple, well-defined operations. The REST architectural style constrains an architecture to a client/server architecture and is designed to use a stateless communication protocol, typically HTTP. In the REST architecture style, clients and servers exchange representations of resources by using a standardized interface and protocol."\cite{javaEE6}
%\\\\
REST is an architectural style which is based on web-standards and the HTTP protocol. REST was first described by Roy Fielding in 2000.In a REST based architecture everything is a resource. A resource is accessed via a common interface based on the HTTP standard methods. In a REST based architecture you typically have a REST server which provides access to the resources and a REST client which accesses and modifies the REST resources. Every resource should support the HTTP common operations. Resources are identified by global IDs (which are typically URIs). REST allows that resources have different representations, e.g., text, XML, JSON etc. The REST client can ask for a specific representation via the HTTP protocol (content negotiation)  \cite{vogella}
\\\\
\textbf{HTTP methods} \\
The PUT, GET, POST and DELETE methods are typical used in REST based architectures.
 \cite{vogella}
\subsection{JAX-RS}
First we decided to use the JAX-RS libary. It works with annotations which seems really easy.
The code could also be generated with IntelliJ but we were not able to find any possibility to deploy it using the IDE and we encountered too many errors. We were also trying an complete tutorial (\cite{vogella}).Still we were not able to solve any of them, we decided to use another language - PHP. An example for an Hello World JAX-RS Webservice can be found in Listing \ref{RFWS}.
\begin{lstlisting}[caption=Restful Webservice, label=RFWS]
@Path("/helloworld")
public class RestfulWebservice {
    @GET
    @Produces("text/plain")
    public String getClichedMessage() {
        return "Hello World!";
    }
}
\end{lstlisting}



%\textbf{Annotations}\\
%Resource identification through URI: A RESTful web service exposes a set of resources that identify the targets of the interaction with its clients. Resources are identified by URIs, which provide a global addressing space for resource and service discovery. See The @Path Annotation and URI Path Templates for more information.
%\\ \\ \\
%Root resource classes are POJOs that are either annotated with @Path or have at least one method annotated with @Path or a request method designator, such as @GET, @PUT, @POST, or @DELETE. Resource methods are methods of a resource class annotated with a request method designator. This section explains how to use JAX-RS to annotate Java classes to create RESTful web services.\cite{javaEE6}
%\\ \\ \\
%Annotation	Description
%@PATH(your\_path)	Sets the path to base URL + /your\_path. The base URL is based on your application name, the servlet and the URL pattern from the web.xml configuration file.
%@POST	Indicates that the following method will answer to an HTTP POST request.
%@GET	Indicates that the following method will answer to an HTTP GET request.
%@PUT	Indicates that the following method will answer to an HTTP PUT request.
%@DELETE	Indicates that the following method will answer to an HTTP DELETE request.
%@Produces(MediaType.TEXT\_PLAIN[, more-types])	@Produces defines which MIME type is delivered by a method annotated with @GET. In the example text ("text/plain") is produced. Other examples would be "application/xml" or "application/json".
%@Consumes(type[, more-types])	@Consumes defines which MIME type is consumed by this method.
%@PathParam	Used to inject values from the URL into a method parameter. This way you inject, for example, the ID of a resource into the method to get the correct object.
%\\ \\ \\






\newpage
\section{Building the SOA Webservice in Java}
Because we were using the IntelliJ IDEA, this was quite easy.
A hello world example with jax-ws can be generated under \textit{new project} -\textgreater \textit{Check Webservice } -\textgreater Check \textit{Generate server sample code}. \\
This client code is a Hello World example and can already be run. When we accessed it trough the webbrowser we were able to see some information about the service and the generated wsdl File.\\
\\
Under \cite{mkyongjaxwsexample} we were able to find a complete example, using an interface and more important also a client. So we were changing our code then. We used the hello world example and the tutorial from mkyong to adapt the following classes:\\
\begin{itemize}
\item \texttt{Searchable}-Interface (Listing \ref{Searchable})
\item \texttt{KnowledgeBaseSearcher}-Class which implements the \texttt{Searchable}-Interface and provides the search method (Listing \ref{KnowledgeBaseSearcher})
\item \texttt{KnowledgeBaseSearcherPublisher}-Class which publishes the \texttt{KnowledgeBaseSearcher}-Service  (Listing \ref{KnowledgeBaseSearcherPublisher})
\item \texttt{KnowledgeBaseSearcherClient}-Class which is using the service over it's WSDL File (Listing \ref{KnowledgeBaseSearcherClient})
\end{itemize}
Man muss nun zuerst den \texttt{KnowledgeBaseSearcherPublisher} starten, um das Service zu publishen und das Starten des Clients (\texttt{KnowledgeBaseSearcherClient}) und das unten angegebene Beispiel nun gibt nun
\textit{Return the search query with the search of: blabla} aus.

\begin{lstlisting}[caption=Searchable interface, label=Searchable]
// imports javax.ws.*
@WebService
@SOAPBinding(style = Style.RPC)
public interface Searchable {
    @WebMethod String search(String searchstring);
}
\end{lstlisting}

\begin{lstlisting}[caption=KnowledgeBaseSearcher class, label=KnowledgeBaseSearcher]
// imports javax.ws.*
@WebService(endpointInterface = "soa.Searchable")
public class KnowledgeBaseSearcher implements Searchable{
    @Override
    public String search(String searchstring) {
        return "Return the search query with the search of: "+searchstring;
    }
\end{lstlisting}
\begin{lstlisting}[caption=KnowledgeBaseSearcherPublisher class, label=KnowledgeBaseSearcherPublisher]
public class KnowledgeBaseSearcherPublisher {
    public static void main(String[] args) {
        Endpoint.publish("http://localhost:9999/soa/searcher", new KnowledgeBaseSearcher());
    }
}
\end{lstlisting}
\begin{lstlisting}[caption=KnowledgeBaseSearcherClient class, label=KnowledgeBaseSearcherClient]
public class KnowledgeBaseSearcherClient {
    public static void main(String[] args) throws Exception {
        URL url = new URL("http://localhost:9999/soa/searcher?wsdl");

        //1st argument service URI, refer to wsdl document above
        //2nd argument is service name, refer to wsdl document above
        QName qname = new QName("http://soa/", "KnowledgeBaseSearcherService");

        Service service = Service.create(url, qname);
        Searchable searcher = service.getPort(Searchable.class);

        System.out.println(searcher.search("blabla"));
    }
}
\end{lstlisting}





\newpage
\begin{thebibliography}{56}

\bibitem{javaEE6}
   \textbf{The Java EE 6 Tutorial} \\
  \textit{http://docs.oracle.com/javaee/6/tutorial/doc/gijqy.html}
  \newline last used: 06.04.2014, 11:52
  
  \bibitem{vogella}
   \textbf{REST with Java (JAX-RS) using Jersey - Tutorial}, Lars Vogel\\
  \textit{  http://www.vogella.com/tutorials/REST/article.html}
  \newline last used: 11.04.2014, 13:58
  
    
  \bibitem{mkyongjaxwsexample}
   \textbf{JAX-WS Hello World Example – RPC Style}, mkyong ,August 29, 2012\\
  \textit{  http://www.mkyong.com/webservices/jax-ws/jax-ws-hello-world-example/}
  \newline last used: 12.04.2014, 10:36
  

  
\end{thebibliography}
\end{document}
